% !TEX root = ./main.tex
%%%%%%%%%%%%%%%%%%%%%%%%%%%%%%%%%%%%%%%%%%%%%%%%%%%%%%%%%%%%%%%%%%%%%%%%%%%%%%%%%%%%%%%%%%
% Dans cette section, introduisez toutes les notations mathématiques que vous jugez      %
% utiles à la réalisation du projet.                                                     %
%%%%%%%%%%%%%%%%%%%%%%%%%%%%%%%%%%%%%%%%%%%%%%%%%%%%%%%%%%%%%%%%%%%%%%%%%%%%%%%%%%%%%%%%%%
\section{Formalisation du Problème}\label{formalisation}
%%%%%%%%%%%%%%%%%%%%%%%%%%%%%%%%%%%

\subsection{Utilisez les bons opérateurs}

Voir la table \ref{table:op}.

\begin{table}[!h]
\centering
\begin{tabular}{l c}
Nom & Op \\
\hline
ET & $\land$ \\
OU & $\lor$ \\
Quantification universelle & $\forall$ \\
Quantification existentielle & $\exists$ \\
\end{tabular}
\caption{Opérateurs les plus usuels en logique}
\label{table:op}
\end{table}

\subsection{Trouver un symbole précis}

Voir ce site : \url{http://detexify.kirelabs.org/classify.html}. Il suffit de dessiner le symbole dont vous avez besoin et le site trouvera (normalement) la bonne commande à taper (ainsi que le package à éventuellement inclure si besoin est).
